\documentclass{beamer}

\title{TODO}
\author{Francisco Figari}
\date{Buenos Aires, 2022}
% https://latex-beamer.com/tutorials/logo-beamer/
\titlegraphic{\includegraphics[width=8em]{logo-fcen.png}}  

\setbeamertemplate{navigation symbols}{}  % para esconder los controles del
                                          % beamer
                                          % https://stackoverflow.com/a/3017066

\begin{document}

\frame{\titlepage}

% quiénes, qué, cuándo, dónde
Juan, Gus, Bruno y yo
estudiando
  la aplicabilidad de eye tracking web
  al diagnóstico de condiciones neuropsicológicas
  a través de análisis clínicos remotos
durante el último año
acá
  en el LIAA
  en el contexto de las Pasantías de Iniciación en la Investigación en Computación


\section{Introducción}

% más que la intro de acá abajo empezar desarmando el título, que fue cambiando
% constantemente a lo largo del proyecto

% por qué

Los ojos tienen un rol central en expresar emociones y procesos cognitivos por
los cuales una persona está pasando.
Estudiar la mirada de un individuo frente a distintos estímulos nos da entonces
información sobre dónde pone los focos de atención y de manera más general
permite estimar el comportamiento del individuo.
El eye tracking permite realizar estos estudios de manera no invasiva.
Habitualmente para esto se realizan experimentos presenciales con eye trackers
especializados que usualmente tienen costo alto.
  % imgs de algun eyetracker profesional y de otro comercial
  % qué costos, fijarse cuánto salen, si no ewstá el paper de PupilEXT que
  % tiene algunos datos sobre esto
Sustentado en una creciente sofisticación de tecnologías web,
  % de cuándo son los estandares para usar la webcam en los navegadores?
surge interés en aprovechar las cámaras web de las notebooks domésticas.
Además de disminuir los costos, esto permitiría distribuir remotamente los
experimentos.
El eye tracking puede utilizarse además para estudiar la usabilidad en
interfaces web, para detectar pérdidas de atención y para estimar el campo
visual.
  % representar esto con algunas imgs (screen de papers o fotos de cosas)

% cómo
Para entender la aplicabilidad nos propusimos a construir un prototipo de eye
tracker web orientado a realizar análisis clínicos.

comentar diferencias entre ambos contextos

modelado de la mirada
  % modelado explícito
  % estructuras de los ojos
  % modelado por apariencia

estudio de las implementaciones de trabajos previos de eye tracking web para
ver si aplicaban a nuestro objetivo, o si al menos podía reutilizarse alguna 
parte
  % entrar un poco más en detalle en WG, TG y PACE
  % imgs con los titulos de los papers

tarea de antisacadas como caso de estudio

protoclo experimental
qué instancias de experimetacníon hicimos
  % concentrarse directamente en la segunda instancia
qué implementación terminamos armando
  % mencionar las desviaciones en las estimaciones porque eso explica porque la
  % calibración se hizo como se hizo

resultados
  % resultados generales replicados
  % no se pudo estudiar efectos de la edad por baja representatibvidad
  
% capaz me convenga armar directo los frames en lugar de estar escribiendo el
% script si al final estoy haciendo medio lo mismo que la del informe


% \section{foo}
% 
% \begin{frame}
% 
% \frametitle{Sample frame title}
% 
% This is some text in the first frame.
% This is some text in the first frame.
% This is some text in the first frame.
% 
% \end{frame}
% 
% \section{baz}
% 
% \begin{frame}
% 
% \frametitle{Sample frame title}
% 
% This is some text in the first frame.
% This is some text in the first frame.
% This is some text in the first frame.
% 
% \begin{center}
% \includegraphics[width=0.5\linewidth]{pipino.png}
% \end{center}
% 
% \end{frame}

\end{document}
