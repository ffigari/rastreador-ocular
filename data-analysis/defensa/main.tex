\documentclass{beamer}

\title{TODO}
\author{Francisco Figari}
\date{Buenos Aires, 2022}
% https://latex-beamer.com/tutorials/logo-beamer/
\titlegraphic{\includegraphics[width=8em]{logo-fcen.png}}  

\setbeamertemplate{navigation symbols}{}  % para esconder los controles del
                                          % beamer
                                          % https://stackoverflow.com/a/3017066

\begin{document}

\frame{\titlepage}

% intro

% quiénes, qué, cuándo, dónde
Juan, Gus, Bruno y yo
estudiando
  la aplicabilidad de eye tracking web
  al diagnóstico de condiciones neuropsicológicas
  a través de análisis clínicos remotos
durante el último año
acá
  en el LIAA
  en el contexto de las Pasantías de Iniciación en la Investigación en Computación

% por qué
Sustentado en una creciente sofisticación de tecnologías web,
  % de cuándo son los estandares para usar la webcam en los navegadores?
surge interés en aprovechar las cámaras web de las notebooks domésticas.
Esto evitaría los costos altos asociados habitualmente a eye trackers
  % qué costos, fijarse cuánto salen los eye trackers comerciales, si no está
  % tmb el paper de PupilEXT que tiene algunos datos sobre esto
facilitando además la distribución de los experimentos que se quisiera realizar
  % TukerGaze y crowdsourcing
  % en gral, recolección de datos
  % posibilidad de realizar remotamente seguimiento de pacientes

% cómo

% \section{foo}
% 
% \begin{frame}
% 
% \frametitle{Sample frame title}
% 
% This is some text in the first frame.
% This is some text in the first frame.
% This is some text in the first frame.
% 
% \end{frame}
% 
% \section{baz}
% 
% \begin{frame}
% 
% \frametitle{Sample frame title}
% 
% This is some text in the first frame.
% This is some text in the first frame.
% This is some text in the first frame.
% 
% \begin{center}
% \includegraphics[width=0.5\linewidth]{pipino.png}
% \end{center}
% 
% \end{frame}

\end{document}
