\documentclass{beamer}

\title{TODO}
\author{Francisco Figari}
\date{Buenos Aires, 2022}
\titlegraphic{\includegraphics[width=8em]{logo-fcen.png}}  

\setbeamertemplate{navigation symbols}{}

\begin{document}

\frame{\titlepage}

\begin{frame}
\frametitle{Eye tracking de laboratorio} % Motivación?

  \begin{itemize}
    \item los ojos como expresión de estados emocionales y procesos cognitivos
    \item forma no invasiva de estudiarlos
    \item aplicaciones en el estudio de interfaces, en la oftalmología
    \item[--] altos costos asociados
    \item[+] posibilidad de usar distribución remota de experimentos
    \item[+] nuevas aplicaciones como en la educación digital para detectar
      pérdidas de atención
  \end{itemize}

  \begin{itemize}
    \item creciente interés en llevar eye tracking al navegador web
    \item[+] utilización de hardware existente
    \item[+] distribución remota de experimentos
    \item[+] nuevas aplicaciones, \eg en la educación digital para detectar
      pérdidas de atención
  \end{itemize}
  
\end{frame}

\begin{frame}
\frametitle{Objetivos}
\end{frame}

\begin{frame}
\frametitle{Caso de estudio: tarea de antisacadas}
\end{frame}

% Screens de los papers
\begin{frame}
\frametitle{Implementaciones recientes}
\end{frame}

% Acá van a ser varias diapos para ir dejando todo claro
\begin{frame}
\frametitle{Contexto web}
\end{frame}

% Comentar primera mejora a WG
\begin{frame}
\frametitle{Protocolo experimental}
\end{frame}

% Comentar segunda y terecera mejora a WG
Implementación
  - WG, puntos a favor y en contra
  - necesidad de notificar descalibraciones + detección de movimiento
  - interfaces lógicas, calibracion + validacion

Código de análisis
  - adelantar resultado de bajas y variables frecuencias, figura frecuencias
  - descarte < 15 Hz, interpolación a 30 Hz
  - normalización
  - filtrados varios
  - detección de sacadas

Resultados
  - elevada proporción de ensayos descartados
  - nula representatividad de sujetos de más de 50 años
  

\end{document}
