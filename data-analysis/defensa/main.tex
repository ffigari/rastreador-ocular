\documentclass[aspectratio=169]{beamer}

\usepackage{subcaption}
\usepackage{emoji}

\title{Evaluación y desarrollo de \textit{eye tracker} remoto en navegadores
\textit{web}}
\author{\makebox[.9\textwidth]{\Large Francisco Figari}\\Juan Kamienkowski, Gustavo Juantorena, Bruno Bianchi}
\date{Buenos Aires, 2022}
\titlegraphic{\includegraphics[width=8em]{img/logo-fcen.png}}

\setbeamertemplate{navigation symbols}{}

% TODO: Hacer que esto sea opcional
\setbeamertemplate{frametitle}{
  \insertsectionhead\par
  \vspace*{0.2mm}
  \insertframetitle
}

\begin{document}

\frame{\titlepage}

\section{Introducción}

\begin{frame}{~}

  \begin{figure}
    \begin{subfigure}{0.49\textwidth}
      \centering
      \includegraphics[width=\linewidth]{img/eye-link-eeg.jpg}
      \caption{\textit{Eye tracking} combinado con electroencefalograma}
    \end{subfigure}
    \begin{subfigure}{0.49\textwidth}
      \centering
      \includegraphics[width=\linewidth]{img/reading-fixations-saccades.jpg}
      \caption{Estimación de la mirada durante una tarea de lectura}
    \end{subfigure}
  \end{figure}

\end{frame}

\begin{frame}{\textit{Eye tracking} tradicional}

  \begin{columns}
    \begin{column}{0.5\textwidth}
      \begin{figure}
        \centering
        \includegraphics[width=\linewidth]{img/eye-link-chinrest.jpg}
        \caption{La restricción de movimiento facilita mantener calibrado el
        sistema}
      \end{figure}
    \end{column}

    \begin{column}{0.5\textwidth}
      \begin{figure}
        \centering
        \includegraphics[width=0.8\linewidth]{img/tobii-eyetracker.jpg}
        \caption{El costo de \textit{eye trackers} comerciales puede resultar
        prohibitivo}
      \end{figure}
    \end{column}
  \end{columns}
\end{frame}

\begin{frame}{\textit{Eye tracking} en navegadores web}
  \begin{figure}
    \begin{subfigure}{0.49\textwidth}
      \centering
      \includegraphics[width=0.8\linewidth]{img/external-webcam.jpg}
    \end{subfigure}
    \begin{subfigure}{0.49\textwidth}
      \centering
      \includegraphics[width=0.8\linewidth]{img/notebook.jpg}
    \end{subfigure}
    \caption{Webcams domésticas ya disponibles}
  \end{figure}
\end{frame}

\begin{frame}{Implicancias del contexto remoto de navegador \textit{web}}

  \begin{itemize}
    \item[\emoji{thumbs-up}] Posibilidad de reutilizar cámaras web

    \item[\emoji{thumbs-up}] Compatibilidad con otras herramientas web

    \vspace{0.5cm}

    % TODO: Agregar división acá
    \item[\emoji{pinched-fingers}] Necesidad de implementar sobre
      \texttt{JavaScript}

    \vspace{0.5cm}

    % TODO: Agregar división acá
    % no sólo las webcams son variables si no que tmb la compu donde corre el 
    % programa
    \item[\emoji{thumbs-down}] Hardware variable de potencialmente bajo rendimiento

    % deben transmitirse en texto e imagenes sin que puedan hacerse
    % aclaraciones en el momento. esto implica una duración total del
    % experimento potencialmente mayor
    \item[\emoji{thumbs-down}] Instrucciones transmitidas de manera
      asincrónica a través del navegador

    % puede variar la luz o la disposición del hardware
    \item[\emoji{thumbs-down}] Ambiente físico no controlado

    % TODO: Agregar división acá
    \item[\emoji{thumbs-down}] Limitados a una pequeña fracción de la
      bibliografía debido a tener una y sólo una cámara
    \item[\emoji{thumbs-down}] Ausencia de invarianza frente a movimientos de
      cabeza
  \end{itemize}

\end{frame}
\begin{frame}{Otros trabajos}
  \begin{columns}
    \begin{column}{0.5\textwidth}
      \begin{figure}
        \centering
        \includegraphics[width=0.8\linewidth]{img/pupil-ext.png}
        \includegraphics[width=0.8\linewidth]{img/PACE.png}
      \end{figure}
    \end{column}

    \begin{column}{0.5\textwidth}
      \begin{figure}
        \centering
        \includegraphics[width=0.7\linewidth]{img/webgazer.png}
        \includegraphics[width=\linewidth]{img/turker-gaze.png}
      \end{figure}
    \end{column}
  \end{columns}
\end{frame}

\section{Objetivos}

\begin{frame}{~}

  \begin{enumerate}
    \item Evaluar implementaciones existentes
    \item Implementar un prototipo
    \item Validarlo buscando replicar resultados reportados en la bibliografía
  \end{enumerate}
\end{frame}

\section{Implementación}

\begin{frame}{\texttt{WebGazer} como punto de partida}

  \begin{columns}
    \begin{column}{.5\textwidth}
      \begin{itemize}
        \item[\emoji{thumbs-up}] Extracción de \textit{frames} a través de la
          API del navegador
        \item[\emoji{thumbs-up}] Modelos de localización de los ojos y de
          estimación de la mirada
        \item[\emoji{thumbs-down}] Calibración inadecuada
        \item[\emoji{thumbs-down}] Ausencia de notificación de descalibraciones
        \item[\emoji{party-popper}] Corregida falla de \texttt{WebGazer} que
          causaba \textit{crashes} del navegador en ciertas notebooks
      \end{itemize}
    \end{column}

    \begin{column}{.5\textwidth}
      \begin{figure}
        \includegraphics[width=0.75\linewidth]{img/facemesh-kepyoints.jpg}
        \caption{\textit{Output} del modelo de \textit{facemesh} utilizado por
        \texttt{WebGazer} para la localización de los ojos}
      \end{figure}
    \end{column}
  \end{columns}

\end{frame}

\begin{frame}{Calibración y validación}
  \begin{itemize}
    \item Nuestro caso de uso no garantiza interacciones

    \item Se muestran una serie de puntos, para cada uno de los cuales el
      usuario tendrá que fijar la mirada y presionar la barra de espacio
    
    \item Validación post calibración implementada de similar manera

    \item[\emoji{party-popper}] \texttt{WebGazer} adaptado para evitar cómputos
      ya no necesarios
  \end{itemize}
\end{frame}

\begin{frame}{Notificación de descalibración}

  \begin{columns}
    \begin{column}{.5\textwidth}
      \begin{itemize}
        \item Basada en detectar movimiento
        \item Instanciada luego de cada calibración
        \item Verificación realizada para cada \textit{frame}
        \item[\emoji{party-popper}] \texttt{WebGazer} adaptado para exponer los
          recuadros calculados en cada \textit{frame} por la rutina de
          localización de ojos
      \end{itemize}
    \end{column}

    \begin{column}{.5\textwidth}
      \begin{figure}
        \centering
        \includegraphics[width=\textwidth]{img/eyetracker-playground-screenshot.png}
        \caption{Detección de movimiento en funcionamiento}
      \end{figure}
    \end{column}
  \end{columns}

\end{frame}

\section{Experimentación}

\begin{frame}{Caso de estudio: tarea de antisacadas}

  \begin{columns}
    \begin{column}{.5\textwidth}
      \begin{itemize}
        \item Clínicamente relevante
        \item Resultados esperados ya establecidos
        \item Tarea simple para validar movimientos oculares
      \end{itemize}
    \end{column}
    \begin{column}{.5\textwidth}
      \begin{figure}
        \centering
        \includegraphics[width=\linewidth]{img/antisaccades-protocol.png}
        \caption{Protocolo de las tareas}
      \end{figure}
    \end{column}
  \end{columns}

\end{frame}

\begin{frame}{Primera instancia}
  \begin{itemize}
    \item Limitados a 10 minutos debido a una falla de \texttt{WebGazer}
    \item Únicamente ensayos de antisacada
    \item Recalibración luego de cada notificación de descalibración
    \item Sin validación post calibración
  \end{itemize}
\end{frame}

\begin{frame}{Segunda instancia}
  \begin{itemize}
    \item Duración superior a 20 minutos
    \item Ensayos de prosacadas y de antisacadas
    \item Recalibración cada 10 ensayos y sólo si se detectó una
      descalibración
    \item Con validación post calibración
  \end{itemize}
\end{frame}

\begin{frame}{Implementación y distribución}

  \begin{columns}
    \begin{column}{0.4\textwidth}
      \begin{figure}
        \centering
        \includegraphics[width=\textwidth]{img/jspsych-logo.jpg}
      \end{figure}
    \end{column}
    \begin{column}{0.6\textwidth}
      \begin{figure}
        \centering
        \includegraphics[width=0.8\textwidth]{img/cognition-run-logo.png}
        \includegraphics[width=\textwidth]{img/neuropruebas-logo.jpg}
      \end{figure}
    \end{column}
  \end{columns}
\end{frame}


\section{Resultados}

\begin{frame}{Ejemplo de \textit{output} del sistema}
  \begin{figure}
    \centering
    \includegraphics[width=\linewidth]{plots/output-example.png}
  \end{figure}
\end{frame}

\begin{frame}{Frecuencias de muestreo}
  \begin{figure}
    \centering
    \includegraphics[width=\linewidth]{plots/sampling-frequencies-distribution.png}
  \end{figure}
\end{frame}

\begin{frame}{Anchos de pantalla}
  \begin{figure}
    \centering
    \includegraphics[width=\linewidth]{plots/screens-widths-distribution.png}
  \end{figure}
\end{frame}

\begin{frame}{Estimaciones desviadas}
  \begin{figure}
    \centering
    \includegraphics[width=\linewidth]{plots/skewed-estimations-examples.png}
    \caption{Las estimaciones de algunos sujetos están desviadas de los valores reales}
  \end{figure}
\end{frame}

\begin{frame}{Limpieza y normalización}
  \begin{columns}
    \begin{column}{0.4\textwidth}
      \begin{itemize}
        \item Ensayos descartados: \begin{itemize}
          \item frecuencia menor a 15 Hz
          \item a mano
          \item si el sujeto se distrajo de la tarea
        \end{itemize}
        \item Frecuencia de muestreo: \textit{Upsampling} a 30 Hz usando
          interpolación lineal
      \end{itemize}
    \end{column}
    \begin{column}{0.6\textwidth}
      \begin{figure}
        \centering
        \includegraphics[width=\linewidth]{plots/normalization-example.png}
        \caption{Normalizado y espejado}
      \end{figure}
    \end{column}
  \end{columns}
\end{frame}

\begin{frame}{Detección de sacadas}
  \begin{columns}
    \begin{column}{0.5\textwidth}
      \begin{figure}
        \centering
        \includegraphics[width=\linewidth]{img/saccades-example.jpg}
        \caption{Los movimientos oculares muestran qué elementos de una imagen
        capturan la atención}
      \end{figure}
    \end{column}
    \begin{column}{0.5\textwidth}
      \begin{figure}
        \centering
        \includegraphics[width=\linewidth]{plots/detected-saccades-example.png}
        \caption{Sacadas detectadas sobre las estimaciones de un ensayo}
      \end{figure}
    \end{column}
  \end{columns}
\end{frame}

\begin{frame}{Conclusiones generales replicadas}
  \begin{table}
    \centering
    \begin{tabular}{ l | c | c | c }
      & correctitud & \multicolumn{2}{ c }{tiempo de respuesta (ms)} \\
      &             & correcto & incorrecto \\
      \hline
      antisacada & 81.29\% & 509 (93) & 346 (105) \\
    \end{tabular}
    \caption{Primera instancia}
  \end{table}
  
  \begin{table}
    \centering
    \begin{tabular}{ l | c | c | c }
      & correctitud & \multicolumn{2}{ c }{tiempo de respuesta (ms)} \\
      &             & correcto & incorrecto \\
      \hline
      antisacada & 94.82\% & 358 (109) & 299 (103) \\
      \hline
      prosacada & 98.09\% & 320 (108) & 311 (150) \\
    \end{tabular}
    \caption{Segunda instancia}
  \end{table}

  \emoji{thumbs-down} En la bibliografía para la tarea de antisacadas se
  reportan \textbf{valores de correctitud} más cercanos al rango $[60\%,
  75\%]$.
\end{frame}

\begin{frame}{Menos datos de lo esperado}
  \begin{columns}
    \begin{column}{0.4\textwidth}
      \begin{itemize}
        \item Cantidad de ensayos iniciales baja en relación a otros trabajos
        \item Descarte de aproximadamente $\frac{2}{3}$ de los datos
        \item Altas e inesperadas tasas de correctitud
      \end{itemize}
    \end{column}

    \begin{column}{0.6\textwidth}
      \begin{table}
        \centering

        \begin{tabular}{ l | c | c }
          Antisacadas   & incorrecto  & correcto \\
          \hline
          \# total      & 64          & 1173 \\
          \hline
          \# por sujeto & 4.57 (2.84) & 78.20 (40.38)
        \end{tabular}

        \vspace{0.3cm}

        \begin{tabular}{ l | c | c }
          Prosacadas    & incorrecto  & correcto \\
          \hline
          \# total      & 22          & 1134 \\
          \hline
          \# por sujeto & 2.44 (1.23) & 75.59 (38.58)
        \end{tabular}

        \caption{Desbalance entre grupos incorrecto y grupo correcto}
      \end{table}
    \end{column}
  \end{columns}
\end{frame}

\section{Conclusiones}

\begin{frame}{~}
  \begin{itemize}
    \item Campo aún en sus primeros pasos
    \item El prototipo obtenido de \textit{eye tracker} para navegadores web
      permitió replicar resultados generales
    \item Rendimiento muy por debajo de aquellos
    \item Potencial para realizar análisis clínicos remotos
  \end{itemize}
\end{frame}

\begin{frame}{Limitaciones}
  \begin{itemize}
    \item[\emoji{cross-mark}] Tasas de correctitud demasiado altas
    \item[\emoji{eye}] Pestañeos no considerados
    \item[\emoji{magnifying-glass-tilted-right}] Falta de estudio sobre el
      rendimiento del sistema
  \end{itemize}
\end{frame}

\begin{frame}{Trabajo futuro}
  \begin{itemize}
    \item[\emoji{check-mark}] Revisar criterios de filtrado; explorar la
      definición de correctitud en la tarea de antisacada
    \item[\emoji{eyes}] Implementar detección de pestañeos; considerarlo
      durante la calibración y durante los análisis posteriores
    \item[\emoji{magnifying-glass-tilted-left}] Implementar análisis de
      sensibilidad para estudiar el rendimiento; en particular buscar comparar
      contra \textit{eye trackers} profesionales
  \end{itemize}
\end{frame}

\end{document}
