% https://tex.stackexchange.com/a/84401/273585
\newcolumntype{G}{X}
\newcolumntype{M}{>{\hsize=.6\hsize}X}
\newcolumntype{S}{>{\hsize=.25\hsize}X}

\begin{landscape}
\begin{table}
  \caption{Diseños y usos de la tarea de antisacadas}
  \label{tab:antisaccades-designs}
  \centering
    \scriptsize
    \begin{tabularx}{\linewidth}{SMG}
    \toprule
    \textbf{trabajo}  & \textbf{diseño de la tarea} & \textbf{resultados generales obtenidos} \\
    \midrule
    Unsworth \etal \cite{unsworth_2011_distribution_analysis}, buscan aislar procesos cognitivos en sujetos de entre 22 y 50 años utilizando un mecanismo de ajuste de funciones.
    & Los sujetos se dividen mitad y mitad entre antisacada y prosacada. El estímulo visual es un flash rápido de una letra y en lugar de utilizar \eyetracking preguntan luego qué letra apareció. Realizan tres fases de experimentación en las cuales juegan con el tiempo de fijación y el tiempo de \textit{foreperiod} (tiempo entre ensayos). Realizan primero 250 ensayos, luego 350 y finalmente 3500 divididas en 4 días.
    & Buscan aislar mantenimiento de objetivos y resolución de competencia, para lo cual realizan ajuste de funciones, estudiando luego desviaciones y estiramientos de la distribución en función de las distintas variables experimentales. Además, para las prosacadas obtienen una precisión de 0.92 (0.01) y tiempo de respuesta (RT) de 545 ms (26 ms) y para antisacadas 0.62 (0.03) y 734 ms (28 ms). Si aleatorizan el \textit{foreperiod} obtienen resultados constantes para la prosacada pero precisión creciente en [$\sim$0.63, $\sim$0.7] y RT decreciente en [$\sim$800 ms, $\sim$680 ms] al ir de 200 ms a 1400 ms de \textit{foreperiod}. Luego de las 3500 ensayos divididas en 4 días y 14 bloques obtienen desempeño indistinguible entre ambas tareas. \\
    \midrule
    Olincy \etal \cite{olincy_1997_age_diminishes_performance}, estudian impacto de la edad 42 sujetos de entre 19 y 79 años, destacándose el estudio de la memoria espacial.
    & Realizan una tarea visual guiada en la cual deben seguir un estímulo que se mueve 5°, 10° o 15° cada unos instantes y luego una tarea de antisacadas consistente en 1925 ms de fijación, 75 ms de pantalla en blanco, 100 ms de un estímulo lateral (entre 5° y 15°), 700 ms de pantalla en blanco y un punto final donde se supone que estén mirando.
    & Además de RT y tasa de error, reportan el porcentaje de error espacial. Reportan RTs de $\sim$280 ms con tasas de error de $\sim$0.06 para los sujetos de 20 años y $\sim$580 ms con tasas de error de $\sim$0.6 para aquellos de 80 años. No encuentran correlación entre el error espacial y la edad. \\
    \midrule
    Smyrnis \etal \cite{smyrnis_2002_big_sample}, estudio sobre más de 2000 hombres de entre 18 y 25 años.
    & Realizan múltiples estudios entres los cuales está la tarea de antisacadas. Su diseño consiste en un estímulo de fijación durante entre 1 y 2 segundos, sin tiempo en blanco intermedio y con un estímulo lateral a entre 2° y 10°.
    & Buscando definir una base de valores para sujetos neurotípicos, calculan 9 variables distintas entre las cuales están los tiempos de respuesta y las tasas de error. Los resultados son sin embargo conflictivos en que las tasas de error que reportan son similares a tasas de error que otros trabajos obtienen para pacientes esquizofrénicos. \\
    \midrule
    P{\l}omecka \etal \cite{plomecka_2020_retest_reliability}, estudio del efecto de la edad en el control inhibitorio, realizado sobre dos grupos etarios (20-35 años y 60-80 años).
    & Su protocolo sigue el \textit{internationally standardised antisaccade protocol} \cite{antoniades_2013_standarized_protocol} que indica 2 bloques de 60 prosacadas y 3 bloques de 40 antisacadas. Los estímulos se presentan a 10° del centro.
    & Además de \textit{saccadic gain} y velocidades pico, para la primera instancia de reportan: \begin{itemize}
        \item grupo joven, prosacada: RT promedio de 268 ms (sd 83 ms), taza de
        error de 1.3\% (sd 1.92\%)
        \item grupo joven, antisacada: RT promedio de 303 ms (sd 88 ms), taza
        de error de 7.83\% (sd 6.54\%)
        \item grupo mayor, prosacada: RT promedio de 309 ms (sd 118 ms), taza
        de error de 5.35\% (sd 5.31\%)
        \item grupo mayor, antisacada: RT promedio de 360 ms (sd 130 ms), taza
        de error de 17.2\% (sd 14.7\%)
    \end{itemize}
    Obtienen luego \textit{good reliability} y \textit{excellent reliability}, según la interpretación de Cicchetti \etal 1994 de los coeficientes de correlación intraclase (ICC). Realizan dos instancias de experimentación para poder estudiar \textit{test retest reliability} para entender luego el potencial de la tarea de antisacadas como marcador clínico de deterioro cognitivo. \\
    \bottomrule
  \end{tabularx}
\end{table}
\end{landscape}



% \begin{itemize}


%   \item \begin{enumerate}
%     \item \textbf{autores}:
%       Unsworth \etal \cite{unsworth_2011_distribution_analysis}, buscan
%       aislar procesos cognitivos en sujetos de entre 22 y 50 años utilizando un
%       mecanismo de ajuste de funciones.
%     \item \textbf{diseño de la tarea}:
%       Los sujetos se dividen mitad y mitad entre antisacada y prosacada.
%       El estímulo visual es un flash rápido de una letra y en lugar de utilizar
%       \eyetracking preguntan luego qué letra apareció.
%       Realizan tres fases de experimentación en las cuales juegan con el tiempo
%       de fijación y el tiempo de \textit{foreperiod} (tiempo entre ensayos).
%       Realizan primero 250 ensayos, luego 350 y finalmente 3500 divididas en 4
%       días.
%     \item \textbf{resultados generales obtenidos}:
%       Buscan aislar mantenimiento de objetivos y resolución de competencia, para lo cual realizan ajuste de funciones, estudiando luego
%       desviaciones y estiramientos de la distribución en función de las
%       distintas variables experimentales.
%       Además, para las prosacadas obtienen una precisión de 0.92 (0.01) y
%       tiempo de respuesta (RT) de 545 ms (26 ms) y para antisacadas 0.62 (0.03)
%       y 734 ms (28 ms).
%       Si aleatorizan el \textit{foreperiod} obtienen resultados constantes
%       para la prosacada pero precisión creciente en [$\sim$0.63, $\sim$0.7] y
%       RT decreciente en [$\sim$800 ms, $\sim$680 ms] al ir de 200 ms a 1400 ms
%       de \textit{foreperiod}.
%       Luego de las 3500 ensayos divididas en 4 días y 14 bloques obtienen
%       desempeño indistinguible entre ambas tareas.
%   \end{enumerate}

%   \item \begin{enumerate}
%     \item \textbf{autores}:
%       Olincy \etal \cite{olincy_1997_age_diminishes_performance}, estudian
%       impacto de la edad 42 sujetos de entre 19 y 79 años, destacándose el
%       estudio de la memoria espacial.
%     \item \textbf{diseño de la tarea}:
%       Realizan una tarea visual guiada en la cual deben seguir un estímulo que
%       se mueve 5°, 10° o 15° cada unos instantes y luego una tarea de
%       antisacadas consistente en 1925 ms de fijación, 75 ms de pantalla en
%       blanco, 100 ms de un estímulo lateral (entre 5° y 15°), 700 ms de
%       pantalla en blanco y un punto final donde se supone que estén mirando.
%     \item \textbf{resultados generales obtenidos}:
%       Además de RT y tasa de error, reportan el porcentaje de error espacial.
%       Reportan RTs de $\sim$280 ms con tasas de error de $\sim$0.06 para los
%       sujetos de 20 años y $\sim$580 ms con tasas de error de $\sim$0.6 para
%       aquellos de 80 años.
%       No encuentran correlación entre el error espacial y la edad.
%   \end{enumerate}

%   \item \begin{enumerate}
%     \item \textbf{autores}:
%       Smyrnis \etal \cite{smyrnis_2002_big_sample}, estudio sobre más de 2000
%       hombres de entre 18 y 25 años.
%     \item \textbf{diseño de la tarea}:
%       Realizan múltiples estudios entres los cuales está la tarea de
%       antisacadas.
%       Su diseño consiste en un estímulo de fijación durante entre 1 y 2
%       segundos, sin tiempo en blanco intermedio y con un estímulo lateral a
%       entre 2° y 10°.
%     \item \textbf{resultados generales obtenidos}:
%       Buscando definir una base de valores para sujetos neurotípicos, calculan
%       9 variables distintas entre las cuales están los tiempos de respuesta y
%       las tasas de error.
%       Los resultados son sin embargo conflictivos en que las tasas de error que
%       reportan son similares a tasas de error que otros trabajos obtienen para
%       pacientes esquizofrénicos.
%   \end{enumerate}

%   \item \begin{enumerate}
%     \item \textbf{autores}:
%       P{\l}omecka \etal \cite{plomecka_2020_retest_reliability}, estudio
%       del efecto de la edad en el control inhibitorio, realizado sobre dos
%       grupos etarios (20-35 años y 60-80 años).
%     \item \textbf{diseño de la tarea}:
%       Su protocolo sigue el \textit{internationally standardised antisaccade
%       protocol} \cite{antoniades_2013_standarized_protocol} que indica 2
%       bloques de 60 prosacadas y 3 bloques de 40 antisacadas.
%       Los estímulos se presentan a 10° del centro.
%     \item \textbf{resultados generales obtenidos}:
%       Además de \textit{saccadic gain} y velocidades pico, para la primera
%       instancia de reportan: \begin{itemize}
%         \item grupo joven, prosacada: RT promedio de 268 ms (sd 83 ms), taza de
%         error de 1.3\% (sd 1.92\%)
%         \item grupo joven, antisacada: RT promedio de 303 ms (sd 88 ms), taza
%         de error de 7.83\% (sd 6.54\%)
%         \item grupo mayor, prosacada: RT promedio de 309 ms (sd 118 ms), taza
%         de error de 5.35\% (sd 5.31\%)
%         \item grupo mayor, antisacada: RT promedio de 360 ms (sd 130 ms), taza
%         de error de 17.2\% (sd 14.7\%)
%       \end{itemize}
%       Obtienen luego \textit{good reliability} y \textit{excellent
%       reliability}, según la interpretación de Cicchetti \etal 1994 de los
%       coeficientes de correlación intraclase (ICC).
%       Realizan dos instancias de experimentación para poder estudiar
%       \textit{test retest reliability} para entender luego el potencial de la
%       tarea de antisacadas como marcador clínico de deterioro cognitivo.
%   \end{enumerate}

% \end{itemize}
