\section{{Resultados}}

Como se mencionó anteriormente, se realizaron dos instancias de experimentación.
En una primera instancia se consideraron únicamente antisacadas (N =
{starting_sample_count_stats__subjects_count___first} sujetos) y en una segunda
instancia se incluyeron tanto antisacadas como prosacadas (N =
{starting_sample_count_stats__subjects_count___second} sujetos).

\footnote{{asd; \textit{{italic}}}}
\subsection{{Primera instancia}}


\begin{{figure}}
    \centering
    \includegraphics[width=0.4\linewidth]{{media/calibration-stimulus-left.png}}
    \includegraphics[width=0.4\linewidth]{{media/calibration-stimulus-center.png}}
    \includegraphics[width=0.4\linewidth]{{media/calibration-stimulus-right.png}}
    \caption{{Estímulos de calibración}}
    Al realizar una calibración asistida el sujeto es presentado con estímulos visuales en las regiones de interés.
    Para avanzar con la calibración, el sujeto deberá fijar la mirada en cada estímulo que aparece y luego presionar la barra de espacio.
    \label{{fig:calibration_stimulus}}
\end{{figure}}
