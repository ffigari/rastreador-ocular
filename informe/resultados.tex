\section{{Resultados}}

Como se mencionó anteriormente, se realizaron dos instancias de experimentación.
En una primera instancia se consideraron únicamente antisacadas (N =
{starting_sample_count_stats__subjects_count___first} sujetos) y en una segunda
instancia se incluyeron tanto antisacadas como prosacadas (N =
{starting_sample_count_stats__subjects_count___second} sujetos).

\subsection{{Primera instancia}}

En la primera instancia se realizaron únicamente antisacadas. Se obtuvieron un
total de {starting_sample_count_stats__trials_count___first} ensayos
provenientes de {starting_sample_count_stats__subjects_count___first} sujetos.

# TODO: Figura distribución de las edades

# TODO: Figura antisacadas desagregadas

\subsubsection{{Estimaciones desviadas}}

# TODO: Mover los valores citados en la prosa a dentro de las figuras
        Dsp usar la figura como cita al hacer una afirmación
        Esto va a lograr que naturalmente los valores citados sean valores
        calculables

Se encontró que para varios sujetos las estimaciones obtenidas por el prototipo
sobre el eje horizontal no coincidían con las posiciones reales de la mirada.
En cambio para cada uno de estos sujetos se detectó una desviación de sus
estimaciones durante todo el experimento.
La Figura \ref{fig:skewed-estimations-example} ilustra este fenómeno:
durante la fase de fijación para los sujetos 47 y 24 se obtienen estimaciones
respectivamente cercanas a los valores 2100 y 1400 píxeles, cuando los valores
reales serían 1100 y 900.
Esto no ocurre para todos los sujetos, como puede verse con los sujetos 43 y 22.

Para cada sujeto las estimaciones obtenidas son sin embargo consistentes en
cuánto a su desviación. Si bien los valores obtenidos no coinciden con las
coordenadas reales de los estímulos, sí se mantendrá el posicionamiento
relativo de las estimaciones durante la duración del experimento. Entonces, es
posible continuar identificando las regiones de interés (izquierda, centro y
derecha) del experimento. Estas desviaciones tuvieron que tenerse en cuenta al
momento de normalizar los datos. Asimismo, el hecho que se mantuviera el
correcto posicionamiento relativo dió lugar al mecanismo de validación
implementado.

\subsubsection{{Tiempos de respuesta}}

De los ensayos con respuesta, 555 obtuvieron respuesta correcta y 139 respuesta
incorrecta.
103 de las 139 respuestas incorrectas tuvieron una sacada correctiva antes del
fin del ensayo, es decir que tras mirar al punto incorrecto, corrigió y realizó
una sacada hacia el punto correcto.
Los ensayos correctos tuvieron un tiempo de respuesta mayor al de las
incorrectas (promedio ± desvio std: Correctas = 484 ms ± 118 ms; Incorrectas =
356 ms ± 114 ms).



examples below here
\footnote{{asd; \textit{{italic}}}}
\begin{{figure}}
    \centering
    \includegraphics[width=0.4\linewidth]{{media/calibration-stimulus-left.png}}
    \includegraphics[width=0.4\linewidth]{{media/calibration-stimulus-center.png}}
    \includegraphics[width=0.4\linewidth]{{media/calibration-stimulus-right.png}}
    \caption{{Estímulos de calibración}}
    Al realizar una calibración asistida el sujeto es presentado con estímulos visuales en las regiones de interés.
    Para avanzar con la calibración, el sujeto deberá fijar la mirada en cada estímulo que aparece y luego presionar la barra de espacio.
    \label{{fig:calibration_stimulus}}
\end{{figure}}
