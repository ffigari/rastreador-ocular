\section{{Resultados}}

En ambas instancias luego de la limpieza de los datos se finalizó con menos de
16 sujetos, descartando en el proceso aproximadamente 2 tercios de los ensayos
realizados (tabla \ref{{tab:clean-up-results}}).
Las figuras \ref{{fig:first-starting-sample-distribution}} y
\ref{{fig:second-starting-sample-distribution}} muestran la distribución de
ensayos según edad, frecuencia de muestreo y anchos de pantalla.
Se destaca la ausencia de sujetos de edad mayor a 50 años luego de aplicar los
criterios de filtrado.

TODO: Figura de frecuencia de muestreo en función de la edad

\begin{{table}}
  \centering
  \begin{{tabular}}{{ c c | c | c }}
    \multicolumn{{2}}{{c}}{{ronda}} & primera & segunda \\ 
    \hline
    \multirow{{2}}{{10em}}{{pre-limpieza}}
      & sujetos
      & {first__starting_sample__subjects_count}
      & {second__starting_sample__subjects_count} \\  
      & ensayos
      & {first__starting_sample__trials_count}
      & {second__starting_sample__trials_count} \\  
    \hline
    \multirow{{2}}{{10em}}{{post-limpieza}}
      & sujetos
      & {first__inlier_sample__subjects_count}
      & {second__inlier_sample__subjects_count} \\  
      & ensayos
      & {first__inlier_sample__trials_count}
      & {second__inlier_sample__trials_count} \\  
    \hline
    \multirow{{2}}{{10em}}{{proporción conservada}}
      & sujetos
      & {first__kept_subjects_percentage}\%
      & {second__kept_subjects_percentage}\% \\  
      & ensayos
      & {first__kept_trials_percentage}\%
      & {second__kept_trials_percentage}\% \\  
  \end{{tabular}}

  \caption{{Ensayos pre y post limpieza}}
  \label{{tab:clean-up-results}}
\end{{table}}

\begin{{figure}}
  \centering

  \includegraphics[width=0.8\linewidth]{{results/first-ages-distribution.png}}

  \includegraphics[width=0.8\linewidth]{{results/first-widths-distribution.png}}

  \includegraphics[width=0.8\linewidth]{{results/first-sampling-frequencies-distribution.png}}

  \caption{{Descripción general (primera instancia)}}
  \label{{fig:first-starting-sample-distribution}}
\end{{figure}}

\begin{{figure}}
  \centering

  \includegraphics[width=0.8\linewidth]{{results/second-ages-distribution.png}}

  \includegraphics[width=0.8\linewidth]{{results/second-widths-distribution.png}}

  \includegraphics[width=0.8\linewidth]{{results/second-sampling-frequencies-distribution.png}}

  \caption{{Descripción general (segunda instancia)}}
  \label{{fig:second-starting-sample-distribution}}
\end{{figure}}

Pudieron replicarse resultados generales esperados en la tarea de antisacadas.
Las antisacadas incorrectas mostraron tiempos de respuesta menores que las
antisacadas correctas.
Esto es consistente con la noción de que realizar correctamente la tarea
implica un costo cognitivo adicional.
En la primera instancia de experimentación se obtuvo, para la tarea de
antisacadas, una tasa de correctitud dentro del rango esperable.
Mientras que en la segunda fue mayor, pero sí se comportó como era esperado ya
que fue mayor para el caso prosacada que para antisacada.
Las tablas \ref{{tab:correcteness-rates}} y \ref{{tab:response-times}} así como
las figuras \ref{{fig:first-response-times-distribution}} y
\ref{{fig:second-response-times-distribution}} muestran el detalle de las
correctitudes y tiempos de respuesta obtenidos.
Las figuras \ref{{fig:first-disaggregated-prosaccades}},
\ref{{fig:second-disaggregated-antisaccades}} y
\ref{{fig:second-disaggregated-prosaccades}} muestran los ensayos obtenidos
clasificados según correctitud y tiempo de respuesta.

TODO: Sacadas correctivas

De las respuestas incorrectas, {first__corrected_sample__trials_count} tuvieron
una segunda sacada correctiva antes del fin del ensayo, es decir que tras mirar
al punto incorrecto, corrigió y realizó una sacada hacia el punto correcto.

En un {second__antisaccades_correction_percentage}$\%$ de las antisacadas
incorrectos se realizó una corrección.
Tal corrección tomó en promedio
{second__corrected_antisaccades_sample__mean_correction_delay} ms (stdev
{second__corrected_antisaccades_sample__stdev_correction_delay} ms).

\begin{{table}}
  \centering
  \begin{{tabular}}{{c | c | c c}}
    ronda
      & primera
      & \multicolumn{{2}}{{c}}{{segunda}} \\
    tarea
      & antisacada
      & antisacada
      & prosacada \\
    \hline
    tasa de correctitud
      & {first__antisaccades_correctness_percentage}\%
      & {second__antisaccades_correctness_percentage}\%
      & {second__prosaccades_correctness_percentage}\% \\
  \end{{tabular}}
  \caption{{Tasas de correctitud}}
  \label{{tab:correcteness-rates}}
\end{{table}}

\begin{{table}}
  \centering
  \begin{{tabular}}{{c|cc|cccc}}
    ronda
      & \multicolumn{{2}}{{|c|}}{{primera}}
      & \multicolumn{{4}}{{|c|}}{{segunda}} \\
    tarea
      & \multicolumn{{2}}{{|c|}}{{antisacada}}
      & \multicolumn{{2}}{{|c }}{{antisacada}}
      & \multicolumn{{2}}{{ c}}{{prosacada}} \\
    correctitud
     & correcto & incorrecto
     & correcto & incorrecto
     & correcto & incorrecto \\
    \hline
    tiempo de respuesta en
     &  {first__correct_antisaccades_sample__mean_response_time}
       ({first__correct_antisaccades_sample__stdev_response_time})
     &  {first__incorrect_antisaccades_sample__mean_response_time}
       ({first__incorrect_antisaccades_sample__stdev_response_time})
     &  {second__correct_antisaccades_sample__mean_response_time}
       ({second__correct_antisaccades_sample__stdev_response_time})
     &  {second__incorrect_antisaccades_sample__mean_response_time}
       ({second__incorrect_antisaccades_sample__stdev_response_time})
     &  {second__correct_prosaccades_sample__mean_response_time}
       ({second__correct_prosaccades_sample__stdev_response_time})
     &  {second__incorrect_prosaccades_sample__mean_response_time}
       ({second__incorrect_prosaccades_sample__stdev_response_time}) \\
    ms, promedio (desvio std) & & & & & & \\
  \end{{tabular}}
  \caption{{Tiempos de respuesta}}
  \label{{tab:response-times}}
\end{{table}}

\begin{{figure}}
  \includegraphics[width=\linewidth]{{results/first-response-times-distribution.png}}
  \caption{{Distribución de tiempos de respuesta (primera instancia)}}
  \label{{fig:first-response-times-distribution}}
\end{{figure}}

\begin{{figure}}
  \includegraphics[width=\linewidth]{{results/first-disaggregated-antisaccades.png}}
  \caption{{Antisacadas desagregadas (primera instancia)}}
  \label{{fig:first-disaggregated-prosaccades}}
\end{{figure}}

\begin{{figure}}
  \includegraphics[width=\linewidth]{{results/second-response-times-distribution.png}}
  \caption{{Distribución de tiempos de respuesta (segunda instancia)}}
  \label{{fig:second-response-times-distribution}}
\end{{figure}}

\begin{{figure}}
  \centering
  \includegraphics[width=\linewidth]{{results/second-disaggregated-antisaccades.png}}
  \caption{{Antisacadas desagregadas (segunda instancia)}}
  \label{{fig:second-disaggregated-antisaccades}}
\end{{figure}}

\begin{{figure}}
  \includegraphics[width=\linewidth]{{results/second-disaggregated-prosaccades.png}}
  \caption{{Prosacadas desagregadas (segunda instancia)}}
  \label{{fig:second-disaggregated-prosaccades}}
\end{{figure}}


TODO: Algún comentario más concreto sobre las desviaciones?

TODO: Alguna figura sobre las desviaciones?


TODO: Baja representatividad de grupo incorrecto por sujeto

TODO: Tabla cantidades de correctitud por sujeto de la segunda instancia


{second__correctness_summary_table}

%

TODO: Rever la conclu para que no se repita con esta sección

%%%%%

\subsection{{Primera instancia}}

\subsubsection{{Estimaciones desviadas}} \label{{section:skewed-estimates}}

Se encontró que para varios sujetos [TODO: cuántos sujetos? con qué magnitud?]
las estimaciones obtenidas estaban desviadas del centro (Figura
\ref{{fig:skewed-estimations-example}}).
Para cada sujeto, estas desviaciones fueron consistentes a lo largo de todo el
experimento.
Además se mantuvo la posición relativa de las estimaciones.
Luego de ser apropiadamente normalizados, se logró entonces utilizar los datos
recolectados para identificar si la mirada caía en algunas de las tres regiones
de interés.

\begin{{figure}}
  \centering

  \includegraphics[width=\textwidth]{{results/skewed-estimations.png}}
  Durante la fase de fijación los sujetos 47 y 24 obtuvieron respectivamente
  estimaciones cercanas a los 2100 y 1400 píxeles, cuando los valores reales
  serían 1100 y 900.
  Estas desviaciones no ocurrieron para todo sujeto, como lo muestran los
  sujetos 42 y 22. \\
  El valor de desviación reportado se calcula como la división entre la
  coordenada estimada del centro (el promedio de las estimaciones durante el
  período de fijación) y la coordenada real del centro.

  TODO: Regenerar este plot para que macheen las dimensiones

  \caption{{Ejemplos de sujetos con estimaciones desviadas}}
  \label{{fig:skewed-estimations-example}}
\end{{figure}}

\subsection{{Segunda instancia}}

\begin{{figure}}
  \centering
  \includegraphics[width=\textwidth]{{
    results/second-sampling-frequencies-by-age.png}}
  \caption{{Frecuencia de muestreo en función de la edad}}
  [TODO: \\
  - Regenerar plot y ajustar dimensiones \\
  - Debería escribir algún comentario acá?]
  \label{{fig:sampling-frequencies-by-age}}
\end{{figure}}
