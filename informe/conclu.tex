\chapter{Conclusiones}

El \eyetracking web remoto es un campo en sus primeros pasos de desarrollo pero
que suscita un creciente interés de distintos grupos.
Muchos de estos ya estaban explorando experimentos con \eyetracking web antes
del 2020 pero el aislamiento potenció y realzó su importancia.
Este interés se sustenta en la amplia historia del \eyetracking en el
laboratorio, donde ha mostrado ser una ventana para estudiar procesos
cognitivos de forma no invasiva y de mínimo contacto con el sujeto
experimental.
Nuestra pregunta original puede reformularse como hasta qué punto el
\eyetracking web puede reemplazar o extender a otros contextos el uso de
\eyetrackers tradicionales (en particular aquellos utilizados en laboratorio).
La implementación y experimentación realizadas sugieren un fuerte potencial
pero también muestran múltiples dificultades propias del contexto web remoto.

Como resultado de este trabajo se obtuvo un prototipo de \eyetracker web
utilizable en análisis clínicos remotos que permite obtener conclusiones de
alto nivel sobre el comportamiento del sujeto (\eg, a qué lado de la pantalla
mira el sujeto).
Este puede utilizarse en navegadores web a la par de \jspsych, provee
mecanismos de calibración y validación similares a aquellos de \eyetrackers
comerciales e incluye un mecanismo de detección de descalibraciones.
Se ofrece también la posibilidad de utilizar el prototipo a través de una
interfaz no dependiente de \jspsych.
Además, se proveen criterios de exclusión y normalización para las estimaciones
obtenidas y mecanismos de detección de sacadas.

En el marco del trabajo, se realizó un análisis intuitivo sobre el rendimiento
de la implementación, el cual muestra frecuencias de muestreo menores a los 30
Hz y estimaciones con desviaciones respecto del punto real de la mirada, aunque
con correcto posicionamiento relativo (figura
\ref{fig:skewed-estimations-example}).
Al no estar estandarizado el reporte de errores en las herramientas de
\eyetracking \cite{zandi_2021_pupilext}, se dificulta comparar conclusiones de
distintos análisis clínicos.
Lo mismo ocurre cuando en distintos trabajos se realizan implementaciones
\adhoc de detección de sacadas \cite{salvucci_2000_identifying_fixations}.
En el campo de la oftalmología, para el caso de la perimetría automatizada
\cite{pubmed_1996_automated_perimetry}, aparece también esta dificultad para
comparar resultados obtenidos por distintas herramientas, aunque existan
trabajos que intenten compararlas
\cite{landers_2007_automated_perimeters_comparison}.
La existencia de una herramienta abierta para realizar estudios de movimientos
oculares online eliminaría a las diferencias implementativas como factor de
incomparabilidad.
En base a esto y en líneas con los beneficios mencionados respecto de la
existencia de implementaciones abiertas, se liberaron en un repositorio público
tanto el código del prototipo como aquel de análisis de estimaciones y aquel
utilizado para realizar los experimentos remotos.

\section{Limitaciones}

\subsection{A nivel implementación}

  \begin{itemize}

    \item
      Las estimaciones tuvieron una frecuencia de muestreo promedio por debajo
      de los 30 Hz para todos los sujetos, llegando incluso en algunos casos a
      estar por debajo de los 5 Hz.
      En contraste, para \eyetrackers profesionales de laboratorio se reportan
      frecuencias de muestreo de entre 100 Hz y 2000 Hz
      \cite{hosp_2020_remote_eye}.
      Para el \eyetracker comercial \tobii se reporta una frecuencia de 133 Hz
      \footnote{
        especificaciones de \tobii:
        \url{https://help.tobii.com/hc/en-us/articles/360012483818-Specifications-for-Eye-Tracker-5}
      } y para el \eyelink una frecuencia máxima de 1000 Hz en el modo que
      permite mover la cabeza \footnote{
        especificaciones de \eyelink:
        \url{https://www.sr-research.com/wp-content/uploads/2017/11/eyelink-1000-plus-specifications.pdf}
      }.
      La frecuencia máxima obtenible por la implementación presentada está
      limitada por la frecuencia del monitor utilizado por el sujeto.
      Dado que los monitores modernos comerciales suelen tener al menos 60 Hz,
      esto no explica por sí sólo la baja frecuencia obtenida.
      Estas bajas frecuencias dificultan análisis posteriores y en particular
      la implementación de rutinas de detección de sacadas, lo cual aparenta
      ser un desafío en sí mismo.
  
    \item
      En la implementación actual no se tuvieron en cuenta los pestañeos.
      Por un lado, esto posibilita la entrada de datos inválidos de calibración
      al sistema.
      Podría ocurrir que el sujeto pestañee durante los instantes previos al
      \textit{frame} seleccionado en cada instante de calibración.
      En caso de ocurrir esto, los datos de calibración estarían incluyendo
      incorrectamente un \textit{frame} en el cual los ojos están cerrados,
      causando ruido en el modelo interno de regresión.
      Por otro lado, experimentos informales \footnote{
        hilo de twitter:
        \url{https://twitter.com/_HanZhang_/status/1527762369593606145}
      } muestran como los pestañeos generan efectos negativos en las
      estimaciones.
  
    \item
      No se llegó a combinar el \eyetracking con algún mecanismo de estimación
      de las dimensiones del monitor.
      Por lo tanto, las coordenadas de los estímulos presentados están
      indicadas en píxeles.
      En cambio, en la bibliografía es común utilizar grados de ángulo visual
      para reportar tanto estos valores \cite{munoz_2004_look_away,
      olincy_1997_age_diminishes_performance, smyrnis_2002_big_sample}, como
      errores promedio \cite{huang_2016_pace, santini_2017_eyerectoo}.
      En consecuencia, dos sujetos que cuenten con pantallas de distintos
      tamaños y resoluciones verán los estímulos a distintos ángulos de visión.
      Esto podría ser un problema si un sujeto realiza el experimento en un
      monitor con baja resolución o si está sentado considerablemente cerca de
      la pantalla.
      También podría aumentar la tendencia a mover la cabeza para observar los
      estímulos laterales, facilitando así la descalibración del sistema.
  
    \item
      Al estar el prototipo implementado sobre \js de navegador existe
      variabilidad e imprecisión sobre la duración de cada \textit{frame}.
      Esto es problemático cuando se busca mostrar un estímulo durante una
      corta cantidad de tiempo (< 100 ms).
  
  \end{itemize}

\subsection{A nivel experimentación}

  Se obtuvo una cantidad insuficiente de ensayos por sujeto luego de aplicar
  las rutinas de preprocesamiento sobre los datos obtenidos.
  Además, no se obtuvo suficiente representatividad de distintos grupos
  etarios.
  En consecuencia no pudo estudiarse la asistencia al diagnóstico ni
  buscar replicar resultados relacionados a los efectos de la edad.

  \begin{itemize}

    \item
      Los criterios de exclusión son excesivamente exigentes.
      Mientras que en nuestro caso en ambas instancias se descartó
      aproximadamente 66\% de los ensayos (tabla \ref{tab:clean-up-results}),
      otros trabajos reportaron descartar menos del 2\%
      \cite{unsworth_2011_distribution_analysis}.

    \item
      Las tasas de correctitud obtenidas son demasiado elevadas para la tarea
      de antisacadas.
      En la segunda instancia se obtuvo una tasa de correctitud de 95\%
      mientras que en la bibliografía se reportan valores cercanos al rango
      [60\%, 70\%].
      Esto sugiere que múltiples ensayos incorrectos fueron clasificados como
      correctos lo cual podría estar relacionado a la incapacidad de la rutina
      de detección de sacadas en detectar sacadas pequeñas (figura
      \ref{fig:undetected-saccades-examples}).
      En cierto punto esto es esperable, pues esta rutina incluye como
      reestricción que las sacadas detectadas hayan recorrido una distancia
      mínima de 0.6 unidades post normalización (sección
      \ref{section:saccades-detection}).
      Si bien tal rutina buscó basarse en conceptos mencionados en la
      bibliografía (distancia recorrida, velocidad promedio
      \cite{stuart_2019_saccade_detection_algorithms}) no se midió su eficacia
      ni se comparó sus resultados contra otras implementaciones.
      Como consecuencia de esto es esperable no detectar pequeñas sacadas
      iniciales, detectando en cambio una segunda sacada de mayor largo.
      En particular esto favorecería pasar por alto sacadas incorrectas
      iniciales, causando así tasas de correctitud mayores que las reales.
      Además al estar detectando la segunda sacada realizada, los tiempos de
      respuesta reportados serán mayores a los reales.

      \begin{figure}
        \centering

        \includegraphics[width=\textwidth]{conclu/undetected-saccades-examples.png}

        Se muestran estimaciones obtenidas (en negro) en cuatro ensayos.
        Los parches azules y rojos ilustran las sacadas detectas por la rutina,
        azules siendo aquellas en dirección contraria al estímulo visual y
        rojas aquellas en misma dirección.

        \caption{Sacadas no detectadas}
        \label{fig:undetected-saccades-examples}
      \end{figure}

    \item
      Se obtuvo nula cantidad de sujetos mayores a 50 años
      Una hipótesis para esto es que la forma en la cual se distribuyó el
      experimento no fue adecuada para alcanzar individuos de mayor edad.
      En efecto, este tuvo que realizarse a través de un navegador web y fue
      distribuido a través de redes sociales, lo cual parece más apropiado para
      alcanzar poblaciones de menor edad.
      Sumado a esto, ocurrió que todos los sujetos de entre 50 y 80 años (2
      para la primera instancia y 5 para la segunda instancia) fueron
      descartados durante el preprocesamiento, en particular debido a bajas
      frecuencias de muestreo (figura \ref{fig:sampling-frequencies-by-age}).

  \end{itemize}

\section{Trabajo futuro}

\subsection{Respecto del análisis de datos}

  \begin{itemize}

    \item
      Se deben buscar o desarrollar métodos de detección de sacadas capaces de
      capturar pequeñas sacadas con las restricciones del \eyetracking web.
      Esto evitará que se pasen por alto pequeñas sacadas iniciales.
      Para explorar esto en mayor detalle podrían etiquetarse sacadas en un
      dataset de estimaciones obtenidas con el prototipo implementado.
      De esta manera se obtendría un \groundtruth con el cual podría medirse la
      efectividad de las rutinas que se implemente.
   
    \item
      En el mismo sentido se pueden revisar los criterios de exclusión.
      Por ejemplo, en la implementación actual, para conservar un ensayo es
      requisito que el sujeto haya estado mirando al estímulo central de
      fijación durante la casi totalidad de su aparición.
      Sin embargo no se incluye un mecanismo de detección de fijaciones sino
      que se verifica que durante ese período \begin{enumerate*}
        \item en promedio las estimaciones coincidan con el centro y
        \item no haya sacadas
      \end{enumerate*}.
      Esto se puede mejorar implementando efectivamente mecanismos de detección
      de fijaciones. \\
      La definición de estos criterios impacta en la cantidad de ensayos que
      alcanzan el grupo \inlier así como la calidad de sus estimaciones.
      Debería entonces asegurarse también que estos criterios y
      transformaciones sobre los datos sean adecuados en cuanto al impacto
      sobre las métricas obtenidas (\eg que no ocurra que haya mayor tendencia
      a descartar ensayos con respuesta incorrecta que ensayos con respuesta
      correcta).

  \end{itemize}

  En base a estos puntos, también puede estimarse la tasa de ensayos que el
  criterio descarta y elegirse adecuadamente la cantidad de ensayos que cada
  sujeto debe realizar en futuros experimentos.

  Además, se podría apuntar a conseguir mayor cantidad de sujetos, priorizando
  la representatividad de distintos grupos etarios.
  Con suficiente cantidad de datos se podrían estudiar las diferencias de edad
  utilizando modelos Bayesianos multivariados
  \cite{plomecka_2020_retest_reliability}, modelar los datos obtenidos con el
  modelo SERIA (\textit{Stochastic Early Reaction, Inhibition, and late Action
  model} \cite{aponte_2017_seria}) o realizar análisis de distribuciones sobre
  los tiempos de respuesta \cite{unsworth_2011_distribution_analysis}.
  Una dificultad no evidente de superar en este aspecto es la duración total
  del experimento pues en simultáneo debe maximizarse la cantidad de ensayos
  por experimento y minimizarse su duración.
  Cabe destacar cómo en la segunda instancia la mitad de los sujetos optó por
  cortar tempranamente el experimento.

\subsection{Respecto del desarrollo del prototipo presentado}

  \begin{itemize}
    \item
      Se puede optimizar el prototipo tal que aumente la frecuencia de
      muestreo.
      Para esto tienen que explorarse implementaciones alternativas de sus
      rutinas:
      \begin{itemize}

        \item
          En el prototipo actual, la localización de los ojos depende de
          generar un \facemesh por cada \textit{frame}, lo cual parece
          excesivo pues luego se utiliza una fracción de tal información.
          Sin embargo, al existir métodos específicos a tal problema
          \cite{hansen_2009_eye_of_the_beholder} sería esperable poder realizar
          implementaciones con mejor desempeño.
  
        \item
          Se podría realizar \textit{profiling} de la implementación presentada
          para entender dónde están los cuellos de botella causantes de la
          baja frecuencia. 

  
        \item
          También puede revisarse cómo se capturan \textit{frames} de la webcam
          a través de un navegador web.
          La implementación actual se basa en el uso de \raf pero el objetivo
          real de esta utilidad es sincronizarse con el refresco del monitor
          para realizar animaciones, lo cual no es nuestro caso de uso.
          De existir una forma alternativa de conectarse desde el navegador a
          la webcam se podría desligar la implementación de tal utilidad.
          De esta manera el techo teórico de frecuencia de estimaciones estaría
          dado por la frecuencia de muestreo de la webcam en lugar de por la
          tasa de refresco del monitor.

      \end{itemize}
  
    \item
      Se deberá realizar experimentación donde se estudie la precisión
      obtenible por el prototipo.
      Entre otros, puede estudiarse \begin{enumerate*}
        \item qué tan estáticas son las estimaciones mientras se sabe que el
          sujeto realiza una fijación,
        \item cuánto retraso existe entre que un sujeto reacciona y su registro
          en las estimaciones o
        \item de qué magnitud son y a qué se deben las desviaciones encontradas
          en la primera instancia de experimentación
      \end{enumerate*}.
      En el mismo sentido, durante una misma sesión de experimentación pueden
      compararse las estimaciones del prototipo con estimaciones de
      \eyetrackers profesionales, como lo han hecho en otros trabajos de
      \eyetracking web \cite{xu_2015_turker_gaze, huang_2016_pace}.
      Con una mayor comprensión de las precisiones alcanzables podrá luego
      estudiarse qué información es extrapolable con las estimaciones.
  
    \item
      Otro aspecto del prototipo a explorar con mayor detalle es la detección
      de descalibraciones.
      Una primera opción es explorar modificaciones sobre la detección de
      movimiento en la cual esta esta basada.
      Por ejemplo, puede modificarse el factor de cambio de tamaño de los
      recuadros del ojo (figura \ref{fig:features-to-stillness-region}) para
      estudiar luego su impacto en la notificación de descalibraciones.
      Es esperable también poder detectar descalibraciones sin recurrir a
      detección de movimiento.
      Por ejemplo, \eyetrackers comerciales muestran regularmente un punto para
      detectar si la estimación en ese momento coincide con las coordenadas del
      punto (\eg, \textit{drift correction} en el \eyetracker \eyelink).
      Al momento de implementar alguno de estos mecanismos debe tenerse en
      cuenta su rendimiento y cómo se vea afectada la duración total del
      experimento. \\
      Al margen de cómo se implemente la detección de descalibración, debe
      definirse con mayor claridad qué significa que la herramienta esté
      correctamente calibrada.
      Como ha mostrado la experimentación realizada, es posible que las
      estimaciones arrojadas por la herramienta no coincidan con la mirada real
      pero que aun así pueda extrapolarse información.
      La definición de calibración está entonces ligada a la pregunta que se
      busca responder.
      Si como en nuestro caso alcanza con poder discernir la posición relativa
      de las estimaciones a lo largo del tiempo, entonces sería esperable poder
      relajar el criterio de descalibración actual.
      Similarmente, en el concepto de calibración debería considerarse la
      granularidad espacial deseada para las estimaciones.
      En nuestro caso alcanzó con distinguir tres regiones de interés (centro,
      izquierda y derecha) pero en otros problemas (\eg, rastreo de la atención
      durante una tarea de lectura) serán deseables mayores niveles de
      precisión.
  
    \item
      Es esperable poder encontrar rutinas más sofisticadas para la calibración
      que la rutina actual.
      Como se ha detallado en la introducción, la implementación de esta
      implica múltiples pequeñas decisiones de diseño.
      Para cada una de ellas aplica probar alternativas y estudiar cómo esto
      afecta el rendimiento.
      Siguiendo las líneas de otros trabajos de \eyetracking web, puede buscar
      capturarse el \textit{frame} de mayor coincidencia entre mirada y
      posición del estímulo presentado \cite{huang_2016_pace} o bien probar
      utilizar varios \textit{frames} por estímulo y descartar aquellos en los
      cuales ocurra un pestañeo \cite{xu_2015_turker_gaze}. 
  
    \item
      La calibración actual podría generalizarse para permitir generar
      estimaciones correctas en regiones de interés más allá de aquellas tres
      relevantes a la tarea de antisacadas.
      La implementación presentada está limitada en no poder ser utilizada para
      estudiar comportamiento sobre el eje vertical.
      Aplica entonces ser capaz de elegir las regiones de interés en función
      del experimento que se desea realizar.
  
    \item
      En futuras implementaciones los pestañeos deben ser continuamente
      detectados.
      Si bien no se ha hecho un análisis detallado al respecto, es esperable
      que estos generen ruido significativo sobre las estimaciones.
      Es necesario entonces reportar los períodos de pestañeos para que estos
      puedan ser considerados en subsecuentes análisis de las estimaciones.
  
    \item
      Respecto de la precisión espacial, en el laboratorio se ha desarrollado
      un \plugin para \jspsych que permite conocer en cada sesión la
      relación entre cantidad de píxeles y cantidad de grados de visión
      (\textit{virtual chinrest} \cite{li_2020_virtual_chinrest}).
      Esto posibilita presentar los estímulos de calibración y aquellos de la
      tarea de antisacadas en función del ángulo de la mirada, que tal como se
      mencionó suele ser la forma habitual de reportar estas distancias en la
      bibliografía.
      La implementación puede extenderse tal que estos estímulos estén
      codificados tanto en grados como en píxeles, utilizando el valor en
      grados cuando tal \plugin esté presente.
  
    \item
      También es importante implementar verificaciones sobre las condiciones
      iniciales del entorno, así como también pedidos de ajustes sobre ella e
      incluso barreras de requerimientos tal que de no ser cumplidas no se
      permita proseguir con el experimento.
      Por ejemplo, en \turkergaze se solicita al sujeto ajustar la iluminación
      del ambiente de ser esta inadecuada.
      Otros posibles objetos de verificación incluyen la resolución de la
      cámara web, la resolución del monitor o el posicionamiento del sujeto en
      relación a la cámara.

  \end{itemize}

\subsection{Respecto del \eyetracking web}

  Al margen de las implementaciones existentes, debe tenerse en cuenta como la
  bibliografía existente no suele aplicar al contexto web.
  Por ejemplo, gran parte de los métodos de modelado de la mirada asumen
  capacidades de hardware y de modificación sobre el entorno que no son
  posibles en este contexto \cite{hansen_2009_eye_of_the_beholder}.
  Antoniades \etal \cite{antoniades_2013_standarized_protocol} han presentado
  un estándar de protocolo para la tarea de antisacadas, pero este sugiere
  frecuencias de muestreo no alcanzables en contextos web.
  Tampoco tienen en consideración la necesidad de mostrar las instrucciones del
  experimento a su inicio en lugar de que estas sean indicadas por un
  experimentador, lo cual es esperable que alargue la duración final del
  experimento. 
  
  En resumen, es necesario revisar el problema de estimación de la mirada de
  principio a fin para entender en qué lugares debe profundizarse la
  bibliografía.
  Esto incluye a los modelos principales de localización de los ojos y de
  modelado de la mirada, pero también a todo mecanismo circundante como son la
  detección de sacadas o las rutinas de calibración.
  En tal búsqueda debe tenerse en cuenta que tales métodos serán implementados
  en \js de navegador, lo cual puede presentar dificultades propias.
  
  Las limitaciones del contexto web encontradas hasta el momento indicarían que
  el \eyetracking web puede ser utilizado en un subconjunto de los problemas
  atacables por el \eyetracking tradicional de laboratorio.
  Sin embargo, su fácil distribución así como la ausencia de costos de hardware
  y potencialmente de software podrían presentar nuevas oportunidades.
  Sumado a un estudio de la precisión y del rendimiento alcanzables, es preciso
  entender en qué categorías de problemas puede utilizarse \eyetracking web.
  Problemas conocidos que requieren \eyetracking como la generación de mapas de
  saliencia \cite{xu_2015_turker_gaze}, el modelado de la búsqueda visual
  \cite{clifton_2016_eye_movements_in_reading} o el estudio de la atención
  durante la lectura \cite{clifton_2016_eye_movements_in_reading} podrían
  combinarse con \crowdsourcing para alcanzar mayor cantidad y
  variabilidad de poblaciones.
